\documentclass[12pt]{article}

\title{Science of pseudoscience}
\author{Durgesh Kumar}
\setlength{\parskip}{0.6em}
\setlength{\parindent}{1.5em}
\begin{document}
\maketitle
\subsection*{What is pseudoscience}
Pseudoscience is any statement or belief that people claimed to be scientifically and factually true.It contains of statement that are not factually correct people still puts effort to try to prove it to be true.it is called Pseudoscience as it means fake science or fake knowledge because people puts biased arguments and practical experiments to prove it to be true.
\subsection*{How it is different from science}
Science is basically the understand about the statement and try to prove it to be true by some mathematical and factual formulation along with some practical experiments.Although it seems to be similar to the pseudoscience but there is difference in the approach.
\par
In science we consider any statement to be false if we get some counter-proof by some experiments or logical derivations.It is also bias-free approach.For example in scientific community "3x+1" problem is not considered to be proven true although we have proved it to be true for almost 300 trillions of digits.
\par
In pseudoscientific approach we doesn't consider our statement to be false even after counter-examples and proofs.In this approach we try to find evidence to support our statement and ignores any arguments that try to disprove it.
\subsection*{Ancient Astronomy}
\subsubsection*{}
Let everyone be honest with the reality of today, \textbf{Artificial intelligence} is a \textbf{\textit{phrase}} that has taken several industries and professions by storm. Is it, however, still a \textbf{\textit{phrase}} that falls on deaf ears, or has it gained wide adoption and momentum?
\\
The term "Artificial Intelligence" has been associated with a variety of things over the years.
\\
\textbf{Healthcare}, E-Commerce, Mathematics, Medicine, and even \textbf{Immortality} are just a few instances. Clearly, the examples cover a large range of human imagination. However, there is one area that is relatively unexplored but equally exciting: \textbf{\textit{Artificial intelligence in Astronomy}}.
\\
\subsection* {AI in Astronomy} 
Understanding the Universe is critical in our quest to understand the origins of humans or life on Earth. And, in our quest to understand and find various answers, we become greedy, resulting in a flood of data that we don't know what to do with.
\\
\\
\textbf{\textit{According to Stanford researcher Andrew Ng, AI's capability is the ability to automate anything that a typical person can perform in less than one second of thought.}}
\\
\\
The next generation of AI astronomers aren't just interested in how this technology can sort data. They're looking into what could be an entirely new mode of scientific discovery, in which artificial intelligence maps out parts of the Universe we've never seen before.
\begin{center}

\\
\textit{A gravitational lens. The nearer red galaxy has bent the light of the more distant blue galaxy around it in a horseshoe shape.}
\end{center}
\\
Although Einstein's theory of general relativity predicted this phenomenon in the 1930s, the first example was discovered in 1979.space is huge, and looking at it takes a long time, especially without today's telescopes. As a result, the search for gravitational lenses has been haphazard so far.
\subsection* {Conclusion}
AI's application in Astrophysics is generating astronomical returns and redefining innovation in the world of Astro-Science, while also assisting in the discovery of some of the universe's greatest mysteries.
\\
Astronomers no longer need to burn the midnight oil straining their eyes to \textbf{detect, classify, and decode} spatial objects or hunt for new planets now that AI is being used to discover galaxies. We now have \textbf{\textit{AI-enabled super-telescopes}} that have their work cut out for them in the twenty-first century, and no one is complaining.


\end{document}
